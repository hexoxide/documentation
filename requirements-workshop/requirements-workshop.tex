\documentclass[conference]{IEEEtran}
\IEEEoverridecommandlockouts
% The preceding line is only needed to identify funding in the first footnote. If that is unneeded, please comment it out.
%\usepackage[backend=bibtex,style=verbose-trad2]{biblatex}
\usepackage{cite}
\usepackage{amsmath,amssymb,amsfonts}
\usepackage{algorithmic}
\usepackage{graphicx}
\usepackage[colorlinks=true,citecolor=blue]{hyperref}
\usepackage{textcomp}
\usepackage{xcolor}
\usepackage{adjustbox}
\usepackage{caption}
\usepackage{booktabs}
\usepackage[flushleft]{threeparttable}
\usepackage{csquotes}
%\usepackage[normalem]{ulem}
%\useunder{\uline}{\ul}{}
%\def\BibTeX{{\rm B\kern-.05em{\sc i\kern-.025em b}\kern-.08em
%    T\kern-.1667em\lower.7ex\hbox{E}\kern-.125emX}}
\begin{document}

\title{Recommendation on Process Integration for Acquiring Stakeholder Requirements According to ISO 29148 with Regards to the CERN Load Balancing Project%\\
%{\footnotesize \textsuperscript{*}Note: Sub-titles are not captured in Xplore and
%should not be used}
%\thanks{Identify applicable funding agency here. If none, delete this.}
}

\author{\IEEEauthorblockN{Corne Kenneth Lukken}
\IEEEauthorblockA{\textit{Faculty Digital Media \& Creative Industry} \\
\textit{Amsterdam University of Applied Sciences}\\
Amsterdam, Netherlands \\
info@dantalion.nl}
}

\maketitle

\section*{INTRODUCTION}
This document will work out a possible implementation of the Stakeholder Requirements Definition Process (SRDP) as defined in ISO 29148\cite{iso29148}, this process will consist of three major parts all to achieve a number of end goals. This specific implementation of the SRDP is tailored to how it could have been best implemented for the European Organization for Nuclear Research (CERN) Load Balancing project during the minor Software for Science (SfS) at the Amsterdam University of Applied Sciences (AUAS). The purpose and desired outcome of ISO 29148 will be shortly described before subdividing the SRDP into three sub processes. The tailored implementation should be an improvement to how the requirements were analyzed during this project, for this reason the methodologies used during the project will be discussed before defining the proposed improvement.

\section{ISO 29148 SUMMARY}
ISO 29148 is an international standard that describes the processes required to perform requirements analysis in a business environment. It has formal definitions for how these processes should be organized as well as how the information gathered from these processes should be documented and maintained. Furthermore the standard describes how to iteratively perform these processes so that they can be implemented in a Agile software development cycle. The focus of this document will be on one of the three major parts of the processes defined in ISO 29148, namely the Stakeholder Requirements Definition Process (SRDP).

\section{STAKEHOLDER REQUIREMENTS DEFINITION PROCESS}
The SRDP consist of three sub processes that are to be performed in order, that is to say the results of one process are used in the next one and so forth. The purpose of this entire process can be best taken from the official standard.

\begin{displayquote}
The purpose of the Stakeholder Requirements Definition Process is to define the requirements for a system that can provide the services needed by users and other stakeholders in a defined environment.
\end{displayquote}

This purpose is achieved by meeting a number specified outcomes, however, the specified outcomes themselves are subject to interpretation. The processes implemented according to ISO 29148 should be specifically designed to minimize the room for these differences in interpretation. The following outcomes are defined as requirements to meet the SRDP:

\begin{itemize}
	\item The required system characteristics and context of use of the product functions and services, and
operational concepts are specified.
	\item The constraints on a system solution are defined.
	\item Traceability of stakeholder requirements to stakeholders and their needs is achieved.
	\item The stakeholder requirements are defined.
	\item Stakeholder requirements for validation are identified.
\end{itemize}

The methods to achieves these outcomes will be tailored to the scientific software development and more specifically the CERN load balancing project. As previously mentioned there are three sub processes to achieve these outcomes which initially have to be performed in order but can subsequently be performed in a arbitrary order. Each sub process contributes in its own way to achieving these outcomes and the processes can be seen as steps in refining information.
\begin{enumerate} 
	\item \textbf{Elicit Stakeholder Requirements} Interaction with stakeholders \& others involved with the project to gather essential information.
	\item \textbf{Define Stakeholder Requirements} Documenting \& categorizing the gathered information.
	\item \textbf{Analyze \& Maintain Stakeholder Requirements} Analyzing \& prioritizing the documented information.
\end{enumerate}

Before specifying the recommended implementation for the CERN load balancing project the tasks of each sub process and what the tasks help achieved are briefly described.

\subsection{ELICIT STAKEHOLDER REQUIREMENTS PPROCESS}

The Elicit Stakeholder Requirements Process (ESRP) consist out of two major tasks were in the first task the stakeholders are identified and in the second task the requirements are identified. It is important that all stakeholders throughout the life cycle of the system are identified and that their roles and responsibilities are clear. ISO 29148 recommends identifying stakeholders in sub-classes were many sub-classes together form one type of stakeholder as a result a cross-section of stakeholders is produced which provides a better picture of the problem. The information required during the second task is not in its final formal form, nevertheless enough information to later define the total set of requirements should be gathered. In order to correctly identify all requirements all potential sources for requirements are identified first. For a list of potential sources ISO 29148 is best referenced as it would provide little to no additional value to reiterate them here. To determine the stakeholders, potential sources and elicit requirements a list of techniques is described, once more listing all these techniques would be of little additional value but the following techniques stood out.

\begin{itemize}
	\item Structured workshops with brainstorming
	\item Technical documentation review
	\item Benchmarking processes and systems
	\item Simulations, prototyping, modeling
\end{itemize}

\subsection{DEFINE STAKEHOLDER REQUIREMENTS PROCESS}

The Define Stakeholder Requirements Process (DSRP) is also divided into tasks, for which DSRP has four. Each tasks is accountable for defining a different part of the requirements. The definitions should account for the different stakeholders and the system its life cycles. The information about this sub process is purposely held short as it will not be the focus of this document.

\begin{enumerate}
	\item \textbf{Define constraints} Unavoidable consequences as a results of agreements, management or technical decisions.
	\item \textbf{Define activity sequences} A representative set of activities for the anticipated operation and support.
	\item \textbf{Define interaction} Interaction between the user and the system.
	\item \textbf{Specify critical qualities} Health \& safety concerns such as operational risks but also includes security concerns such as access permissions.
\end{enumerate}

\subsection{ANALYZE \& MAINTAIN STAKEHOLDER REQUIREMENTS PROCESS}

The last sub process consist of six tasks and is the last process to fully formalize all the requirements information. It focuses on analyzing the information and resolving problems so that the information can be documented and formalized. This sub process also includes tasks to ensure stakeholders have been understood correctly and agree with the documented information. Similar to DSRP this information is purposely held short as it is not the focus of this document.

\begin{enumerate}
	\item \textbf{Analyze all requirements} The analysis includes incomplete or ambiguous requirements.
	\item \textbf{Resolve problems} Define solutions for conflicting or impractical requirements.
	\item \textbf{Stakeholder feedback} Attain agreements on conflicting or impractical solutions.
	\item \textbf{Ensure stakeholder needs} Ensure the proposed requirements meet the needs and goals of the stakeholders.
	\item \textbf{Record requirements} Create the final formalized requirements document.
	\item \textbf{Maintain requirements} Update progress on requirements and when they are changed throughout the system life cycle.
\end{enumerate}

\section{USED REQUIREMENTS METHODOLOGIES}

The following section will begin with a story explaining the events during the project and will overall be written in a less formal format. This story is required to concretely define the approach that was taken, what kind of problems that introduced and how they were overcome.

\subsection{STORY}

Projects at AUAS typically start with an appointed Product Owner (PO) although during this project there was a discussion in the first weeks who was actually going to do this. In the end both proposed PO's were clear stakeholders. In the first meeting with the PO a weekly meeting was scheduled and initial information about previous projects was gathered. The project had seen three stages let by previous teams before the start of this project and as a result three sets of documentation were expected. As an intial step to the project the previous software written in the past were attempted to be be run to validate them on current hardware. The hardware platform used had changed significantly during the different 'life cycles' of the project. It took several weeks of failing to get the software running to reach of to the original developer. With the original developer another attempt was made to get the software running but still it would not work even after several hours. It was determined that the amount of work it would take to get this software running on the new hardware, especially the libraries which the software uses was greatly underestimated. A previous project member had already made efforts to get this software running on this new hardware but unfortunately some of his source code was missing together with documentation on how the porting of the libraries was achieved. This caused the decision to rewrite the software and focus on automated dependency installation to prevent these events from reoccurring in the future. The decision to rewrite the software marks the end of the ESRP stage in this project.

\subsection{SUMMARY}

During the project there was a strong emphasis on a single stakeholder to retrieve information even though several more stakeholders were identified. Furthermore, only two of the four identified stakeholders were ever contacted about the project. The retrieval of information from multiple stakeholders to get a better overall picture had possibly greatly improved if information was retrieved from all four. The decision to rewrite the software had likely be made sooner if there had been more communication with different stakeholders, although it is unlikely that all four stakeholders would be willing to exchange information at least two were more than willing to communicate. During the project it became apparent that sitting down with stakeholders and performing 'hands-on' tasks with the stakeholders leads to rapid information gathering \& exchange.

\section{PROPOSED REQUIREMENTS METHODOLOGIES}

During the project two major stakeholders became immediately apparent and as a result both should have been contacted to gather essential information about the projects requirements and to identify additional stakeholders. The identified set of information would have led to the discovery of more stakeholders as well as provide enough information to identify all life cycles of the software. With the life cycles \& stakeholders identified a stakeholder in each life cycle should be asked about the state of the software at that time or their expectations of the software when the time arrives. The previous versions of the software would be attempted to get running by the project and future versions would be sketched or mocked-up. This would provide enough information about the project and its different life cycles to invite all stakeholders to a workshop which could be feasibly be held in the fourth week of the project.

\subsection{LIFE CYCLES}

Each piece of software has at least three stages in its life cycles, although this typically waterfall based view of software is perhaps outdated in modern agile development approaches it still holds true in some fashion. Software always has stages in which is must mature and always stages in which its usage is gradually reduced until it is no longer maintained. The project has seen three previous iterations and all these iterations belong to the past life cycle. The other life cycles are the current and the future life cycles. This simplification is done because the exact end of this project is unknown and it could see many more future iterations. 

\begin{itemize}
	\item \textbf{Past} Previous iterations of the project.
	\item \textbf{Current} The current iteration of the project.
	\item \textbf{Future} Future iterations of the project.
\end{itemize}

\subsection{WORKSHOP}

The main improvement of the proposed methodology is the workshop to be organized around the fourth week of the project. In this workshop as much stakeholders as possible are gathered from each life cycle to have an organized afternoon of activities in which the current project iterations members gather tremendous amounts of information. To motivate past stakeholders (who have since likely graduated) to attend the workshop will have food \& drinks arranged. Future stakeholders are $2^{nd}$ or $3^{rd}$ years students of the AUAS since they have a high probability of becoming future members of project iterations. The workshop will likely consist between seven to nine attendants and take around four hours. The workshop will consist of five individual parts including a break with food \& drinks. Before the workshop itself invited attendants are asked to perform some preparations, however, attendants will have different preparations depending on their attributed life cycle category.

\begin{itemize}
	\item \textbf{Introduction} Explains the goal of the workshop as well as the expectations of individual attendants.
	\item \textbf{Questions} Attendants are asked question based on their life cycle category or on an individual basis.
	\item \textbf{Break} A short break with food \& drinks. 
	\item \textbf{Hands-on} Existing software \& conceptual models are demonstrated and attendants experiment with their parameters. 
	\item \textbf{Discussion} The hands-on experience is discussed, emphasis should be on software \& models their problems and how to improve them. 
\end{itemize}

Expected types of preparations include the developers of previous iterations of the software to have it working on a machine they bring to the workshop or to provide documentation beforehand on how to do so. For current stakeholders they are expected to have a clear concept of what kind of goals they would like to reach at the end of this iteration. Finally, for future stakeholders they are requested to think about expectations they have in terms of documentation and software if they were to continue a long running project. These expectations along with an overview of activities and the goal of the workshop will be reiterated at the start of the workshop to provide an introduction.
After the introduction the questions prepared for the workshop will be asked in the order of the software its life cycles, at this time other attendants are allowed to ask the organizers questions or ask other attendants questions. The break should start after around one or one and half hours and allows attendants to consume food \& drinks as well as allow attendants to prepare their content for the hands-on demo. The hands-on part of workshop is the highlight of the event the software and models demonstrated should make problems of previous iterations apparent as well as shine a light on new possible improvements. Basic interactivity during the hands-on should be allowed but thorough discussion about which methodologies or implementations are better should be discouraged. In the final part of the workshop the software and models should be discussed in terms of advantages \& disadvantages in which each attendant decides on their proposed best solution.

\section{COMPARISON}
Implementing a workshop as part of the ESRP could improve software development projects which throughout its life cycles go through multiple teams and have varying stakeholders. It allows to more quickly become a functioning team and identify important information in less time, furthermore it allows to retrieve specific information which otherwise might be unavailable. Including potential future development teams as stakeholder allows to identify the needs to continue development more seamlessly in future iterations. The major downside to such an elaborate workshop is that it requires cooperation of past stakeholders which could have since graduated or might no longer be affiliated. Furthermore, gathering information from many stakeholders might complicate correctly prioritizing the goals each of them has in mind. Finally, this workshop approach might be less applicable in projects were each iteration does not have its own new version of software as product.

% The IEEEtran\cite{Huang:2004dg} class file is used to format your paper and style the text. All margins, 
% column widths, line spaces, and text fonts are prescribed; please do not 
% alter them. You may note peculiarities. For example, the head margin
% measures proportionately more than is customary. This measurement 
% and others are deliberate, using specifications that anticipate your paper 
% as one part of the entire proceedings, and not as an independent document. 
% Please do not revise any of the current designations.

% \subsection{Equations}
% Number equations consecutively. To make your 
% equations more compact, you may use the solidus (~/~), the exp function, or 
% appropriate exponents. Italicize Roman symbols for quantities and variables, 
% but not Greek symbols. Use a long dash rather than a hyphen for a minus 
% sign. Punctuate equations with commas or periods when they are part of a 
% sentence, as in:
% \begin{equation}
% a+b=\gamma\label{eq}
% \end{equation}

% Be sure that the 
% symbols in your equation have been defined before or immediately following 
% the equation. Use ``\eqref{eq}'', not ``Eq.~\eqref{eq}'' or ``equation \eqref{eq}'', except at 
% the beginning of a sentence: ``Equation \eqref{eq} is . . .''

% \subsection{\LaTeX-Specific Advice}

% Please use ``soft'' (e.g., \verb|\eqref{Eq}|) cross references instead
% of ``hard'' references (e.g., \verb|(1)|). That will make it possible
% to combine sections, add equations, or change the order of figures or
% citations without having to go through the file line by line.

% Please don't use the \verb|{eqnarray}| equation environment. Use
% \verb|{align}| or \verb|{IEEEeqnarray}| instead. The \verb|{eqnarray}|
% environment leaves unsightly spaces around relation symbols.

% Please note that the \verb|{subequations}| environment in {\LaTeX}
% will increment the main equation counter even when there are no
% equation numbers displayed. If you forget that, you might write an
% article in which the equation numbers skip from (17) to (20), causing
% the copy editors to wonder if you've discovered a new method of
% counting.

% {\BibTeX} does not work by magic. It doesn't get the bibliographic
% data from thin air but from .bib files. If you use {\BibTeX} to produce a
% bibliography you must send the .bib files. 

% {\LaTeX} can't read your mind. If you assign the same label to a
% subsubsection and a table, you might find that Table I has been cross
% referenced as Table IV-B3. 

% {\LaTeX} does not have precognitive abilities. If you put a
% \verb|\label| command before the command that updates the counter it's
% supposed to be using, the label will pick up the last counter to be
% cross referenced instead. In particular, a \verb|\label| command
% should not go before the caption of a figure or a table.

% Do not use \verb|\nonumber| inside the \verb|{array}| environment. It
% will not stop equation numbers inside \verb|{array}| (there won't be
% any anyway) and it might stop a wanted equation number in the
% surrounding equation.

% \subsection{Some Common Mistakes}\label{SCM}
% \begin{itemize}
% \item The word ``data'' is plural, not singular.
% \item The subscript for the permeability of vacuum $\mu_{0}$, and other common scientific constants, is zero with subscript formatting, not a lowercase letter ``o''.
% \item In American English, commas, semicolons, periods, question and exclamation marks are located within quotation marks only when a complete thought or name is cited, such as a title or full quotation. When quotation marks are used, instead of a bold or italic typeface, to highlight a word or phrase, punctuation should appear outside of the quotation marks. A parenthetical phrase or statement at the end of a sentence is punctuated outside of the closing parenthesis (like this). (A parenthetical sentence is punctuated within the parentheses.)
% \item A graph within a graph is an ``inset'', not an ``insert''. The word alternatively is preferred to the word ``alternately'' (unless you really mean something that alternates).
% \item Do not use the word ``essentially'' to mean ``approximately'' or ``effectively''.
% \item In your paper title, if the words ``that uses'' can accurately replace the word ``using'', capitalize the ``u''; if not, keep using lower-cased.
% \item Be aware of the different meanings of the homophones ``affect'' and ``effect'', ``complement'' and ``compliment'', ``discreet'' and ``discrete'', ``principal'' and ``principle''.
% \item Do not confuse ``imply'' and ``infer''.
% \item The prefix ``non'' is not a word; it should be joined to the word it modifies, usually without a hyphen.
% \item There is no period after the ``et'' in the Latin abbreviation ``et al.''.
% \item The abbreviation ``i.e.'' means ``that is'', and the abbreviation ``e.g.'' means ``for example''.
% \end{itemize}
% An excellent style manual for science writers is \cite{b7}.

% \subsection{Authors and Affiliations}
% \textbf{The class file is designed for, but not limited to, six authors.} A 
% minimum of one author is required for all conference articles. Author names 
% should be listed starting from left to right and then moving down to the 
% next line. This is the author sequence that will be used in future citations 
% and by indexing services. Names should not be listed in columns nor group by 
% affiliation. Please keep your affiliations as succinct as possible (for 
% example, do not differentiate among departments of the same organization).

% \subsection{Identify the Headings}
% Headings, or heads, are organizational devices that guide the reader through 
% your paper. There are two types: component heads and text heads.

% Component heads identify the different components of your paper and are not 
% topically subordinate to each other. Examples include Acknowledgments and 
% References and, for these, the correct style to use is ``Heading 5''. Use 
% ``figure caption'' for your Figure captions, and ``table head'' for your 
% table title. Run-in heads, such as ``Abstract'', will require you to apply a 
% style (in this case, italic) in addition to the style provided by the drop 
% down menu to differentiate the head from the text.

% Text heads organize the topics on a relational, hierarchical basis. For 
% example, the paper title is the primary text head because all subsequent 
% material relates and elaborates on this one topic. If there are two or more 
% sub-topics, the next level head (uppercase Roman numerals) should be used 
% and, conversely, if there are not at least two sub-topics, then no subheads 
% should be introduced.

% \subsection{Figures and Tables}
% \paragraph{Positioning Figures and Tables} Place figures and tables at the top and 
% bottom of columns. Avoid placing them in the middle of columns. Large 
% figures and tables may span across both columns. Figure captions should be 
% below the figures; table heads should appear above the tables. Insert 
% figures and tables after they are cited in the text. Use the abbreviation 
% ``Fig.~\ref{fig}'', even at the beginning of a sentence.

% \begin{table}[htbp]
% \caption{Table Type Styles}
% \begin{center}
% \begin{tabular}{|c|c|c|c|}
% \hline
% \textbf{Table}&\multicolumn{3}{|c|}{\textbf{Table Column Head}} \\
% \cline{2-4} 
% \textbf{Head} & \textbf{\textit{Table column subhead}}& \textbf{\textit{Subhead}}& \textbf{\textit{Subhead}} \\
% \hline
% copy& More table copy$^{\mathrm{a}}$& &  \\
% \hline
% \multicolumn{4}{l}{$^{\mathrm{a}}$Sample of a Table footnote.}
% \end{tabular}
% \label{tab1}
% \end{center}
% \end{table}

% \begin{figure}[htbp]
% \centerline{\includegraphics{fig1.png}}
% \caption{Example of a figure caption.}
% \label{fig}
% \end{figure}

% Figure Labels: Use 8 point Times New Roman for Figure labels. Use words 
% rather than symbols or abbreviations when writing Figure axis labels to 
% avoid confusing the reader. As an example, write the quantity 
% ``Magnetization'', or ``Magnetization, M'', not just ``M''. If including 
% units in the label, present them within parentheses. Do not label axes only 
% with units. In the example, write ``Magnetization (A/m)'' or ``Magnetization 
% \{A[m(1)]\}'', not just ``A/m''. Do not label axes with a ratio of 
% quantities and units. For example, write ``Temperature (K)'', not 
% ``Temperature/K''.

% \section*{Acknowledgment}

% The preferred spelling of the word ``acknowledgment'' in America is without 
% an ``e'' after the ``g''. Avoid the stilted expression ``one of us (R. B. 
% G.) thanks $\ldots$''. Instead, try ``R. B. G. thanks$\ldots$''. Put sponsor 
% acknowledgments in the unnumbered footnote on the first page.

\section*{Abbreviations}

\begin{itemize}
	\item \textbf{SRDP} Stakeholder Requirements Definition Process
	\item \textbf{ISO} International Organization for Standardization
	\item \textbf{CERN} European Organization for Nuclear Research
	\item \textbf{SfS} Software for Science
	\item \textbf{AUAS} Amsterdam University of Applied Sciences
	\item \textbf{ESRP} Elicit Stakeholder Requirements Process
	\item \textbf{DSRP} Define Stakeholder Requirements Process
	\item \textbf{AMSRP} Analyze \& Maintain Stakeholder Requirements Process
	\item \textbf{PO} Product Owner
\end{itemize}

\bibliographystyle{IEEEtran}
\bibliography{bibliography}

% \begin{thebibliography}{00}
% \bibitem{b1} G. Eason, B. Noble, and I. N. Sneddon, ``On certain integrals of Lipschitz-Hankel type involving products of Bessel functions,'' Phil. Trans. Roy. Soc. London, vol. A247, pp. 529--551, April 1955.
% \bibitem{b2} J. Clerk Maxwell, A Treatise on Electricity and Magnetism, 3rd ed., vol. 2. Oxford: Clarendon, 1892, pp.68--73.
% \bibitem{b3} I. S. Jacobs and C. P. Bean, ``Fine particles, thin films and exchange anisotropy,'' in Magnetism, vol. III, G. T. Rado and H. Suhl, Eds. New York: Academic, 1963, pp. 271--350.
% \bibitem{b4} K. Elissa, ``Title of paper if known,'' unpublished.
% \bibitem{b5} R. Nicole, ``Title of paper with only first word capitalized,'' J. Name Stand. Abbrev., in press.
% \bibitem{b6} Y. Yorozu, M. Hirano, K. Oka, and Y. Tagawa, ``Electron spectroscopy studies on magneto-optical media and plastic substrate interface,'' IEEE Transl. J. Magn. Japan, vol. 2, pp. 740--741, August 1987 [Digests 9th Annual Conf. Magnetics Japan, p. 301, 1982].
% \bibitem{b7} M. Young, The Technical Writer's Handbook. Mill Valley, CA: University Science, 1989.
% \end{thebibliography}
% \vspace{12pt}
% \color{red}
% IEEE conference templates contain guidance text for composing and formatting conference papers. Please ensure that all template text is removed from your conference paper prior to submission to the conference. Failure to remove the template text from your paper may result in your paper not being published.

\end{document}
